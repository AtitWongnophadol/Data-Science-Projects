
% Default to the notebook output style

    


% Inherit from the specified cell style.




    
\documentclass[11pt]{article}

    
    
    \usepackage[T1]{fontenc}
    % Nicer default font (+ math font) than Computer Modern for most use cases
    \usepackage{mathpazo}

    % Basic figure setup, for now with no caption control since it's done
    % automatically by Pandoc (which extracts ![](path) syntax from Markdown).
    \usepackage{graphicx}
    % We will generate all images so they have a width \maxwidth. This means
    % that they will get their normal width if they fit onto the page, but
    % are scaled down if they would overflow the margins.
    \makeatletter
    \def\maxwidth{\ifdim\Gin@nat@width>\linewidth\linewidth
    \else\Gin@nat@width\fi}
    \makeatother
    \let\Oldincludegraphics\includegraphics
    % Set max figure width to be 80% of text width, for now hardcoded.
    \renewcommand{\includegraphics}[1]{\Oldincludegraphics[width=.8\maxwidth]{#1}}
    % Ensure that by default, figures have no caption (until we provide a
    % proper Figure object with a Caption API and a way to capture that
    % in the conversion process - todo).
    \usepackage{caption}
    \DeclareCaptionLabelFormat{nolabel}{}
    \captionsetup{labelformat=nolabel}

    \usepackage{adjustbox} % Used to constrain images to a maximum size 
    \usepackage{xcolor} % Allow colors to be defined
    \usepackage{enumerate} % Needed for markdown enumerations to work
    \usepackage{geometry} % Used to adjust the document margins
    \usepackage{amsmath} % Equations
    \usepackage{amssymb} % Equations
    \usepackage{textcomp} % defines textquotesingle
    % Hack from http://tex.stackexchange.com/a/47451/13684:
    \AtBeginDocument{%
        \def\PYZsq{\textquotesingle}% Upright quotes in Pygmentized code
    }
    \usepackage{upquote} % Upright quotes for verbatim code
    \usepackage{eurosym} % defines \euro
    \usepackage[mathletters]{ucs} % Extended unicode (utf-8) support
    \usepackage[utf8x]{inputenc} % Allow utf-8 characters in the tex document
    \usepackage{fancyvrb} % verbatim replacement that allows latex
    \usepackage{grffile} % extends the file name processing of package graphics 
                         % to support a larger range 
    % The hyperref package gives us a pdf with properly built
    % internal navigation ('pdf bookmarks' for the table of contents,
    % internal cross-reference links, web links for URLs, etc.)
    \usepackage{hyperref}
    \usepackage{longtable} % longtable support required by pandoc >1.10
    \usepackage{booktabs}  % table support for pandoc > 1.12.2
    \usepackage[inline]{enumitem} % IRkernel/repr support (it uses the enumerate* environment)
    \usepackage[normalem]{ulem} % ulem is needed to support strikethroughs (\sout)
                                % normalem makes italics be italics, not underlines
    

    
    
    % Colors for the hyperref package
    \definecolor{urlcolor}{rgb}{0,.145,.698}
    \definecolor{linkcolor}{rgb}{.71,0.21,0.01}
    \definecolor{citecolor}{rgb}{.12,.54,.11}

    % ANSI colors
    \definecolor{ansi-black}{HTML}{3E424D}
    \definecolor{ansi-black-intense}{HTML}{282C36}
    \definecolor{ansi-red}{HTML}{E75C58}
    \definecolor{ansi-red-intense}{HTML}{B22B31}
    \definecolor{ansi-green}{HTML}{00A250}
    \definecolor{ansi-green-intense}{HTML}{007427}
    \definecolor{ansi-yellow}{HTML}{DDB62B}
    \definecolor{ansi-yellow-intense}{HTML}{B27D12}
    \definecolor{ansi-blue}{HTML}{208FFB}
    \definecolor{ansi-blue-intense}{HTML}{0065CA}
    \definecolor{ansi-magenta}{HTML}{D160C4}
    \definecolor{ansi-magenta-intense}{HTML}{A03196}
    \definecolor{ansi-cyan}{HTML}{60C6C8}
    \definecolor{ansi-cyan-intense}{HTML}{258F8F}
    \definecolor{ansi-white}{HTML}{C5C1B4}
    \definecolor{ansi-white-intense}{HTML}{A1A6B2}

    % commands and environments needed by pandoc snippets
    % extracted from the output of `pandoc -s`
    \providecommand{\tightlist}{%
      \setlength{\itemsep}{0pt}\setlength{\parskip}{0pt}}
    \DefineVerbatimEnvironment{Highlighting}{Verbatim}{commandchars=\\\{\}}
    % Add ',fontsize=\small' for more characters per line
    \newenvironment{Shaded}{}{}
    \newcommand{\KeywordTok}[1]{\textcolor[rgb]{0.00,0.44,0.13}{\textbf{{#1}}}}
    \newcommand{\DataTypeTok}[1]{\textcolor[rgb]{0.56,0.13,0.00}{{#1}}}
    \newcommand{\DecValTok}[1]{\textcolor[rgb]{0.25,0.63,0.44}{{#1}}}
    \newcommand{\BaseNTok}[1]{\textcolor[rgb]{0.25,0.63,0.44}{{#1}}}
    \newcommand{\FloatTok}[1]{\textcolor[rgb]{0.25,0.63,0.44}{{#1}}}
    \newcommand{\CharTok}[1]{\textcolor[rgb]{0.25,0.44,0.63}{{#1}}}
    \newcommand{\StringTok}[1]{\textcolor[rgb]{0.25,0.44,0.63}{{#1}}}
    \newcommand{\CommentTok}[1]{\textcolor[rgb]{0.38,0.63,0.69}{\textit{{#1}}}}
    \newcommand{\OtherTok}[1]{\textcolor[rgb]{0.00,0.44,0.13}{{#1}}}
    \newcommand{\AlertTok}[1]{\textcolor[rgb]{1.00,0.00,0.00}{\textbf{{#1}}}}
    \newcommand{\FunctionTok}[1]{\textcolor[rgb]{0.02,0.16,0.49}{{#1}}}
    \newcommand{\RegionMarkerTok}[1]{{#1}}
    \newcommand{\ErrorTok}[1]{\textcolor[rgb]{1.00,0.00,0.00}{\textbf{{#1}}}}
    \newcommand{\NormalTok}[1]{{#1}}
    
    % Additional commands for more recent versions of Pandoc
    \newcommand{\ConstantTok}[1]{\textcolor[rgb]{0.53,0.00,0.00}{{#1}}}
    \newcommand{\SpecialCharTok}[1]{\textcolor[rgb]{0.25,0.44,0.63}{{#1}}}
    \newcommand{\VerbatimStringTok}[1]{\textcolor[rgb]{0.25,0.44,0.63}{{#1}}}
    \newcommand{\SpecialStringTok}[1]{\textcolor[rgb]{0.73,0.40,0.53}{{#1}}}
    \newcommand{\ImportTok}[1]{{#1}}
    \newcommand{\DocumentationTok}[1]{\textcolor[rgb]{0.73,0.13,0.13}{\textit{{#1}}}}
    \newcommand{\AnnotationTok}[1]{\textcolor[rgb]{0.38,0.63,0.69}{\textbf{\textit{{#1}}}}}
    \newcommand{\CommentVarTok}[1]{\textcolor[rgb]{0.38,0.63,0.69}{\textbf{\textit{{#1}}}}}
    \newcommand{\VariableTok}[1]{\textcolor[rgb]{0.10,0.09,0.49}{{#1}}}
    \newcommand{\ControlFlowTok}[1]{\textcolor[rgb]{0.00,0.44,0.13}{\textbf{{#1}}}}
    \newcommand{\OperatorTok}[1]{\textcolor[rgb]{0.40,0.40,0.40}{{#1}}}
    \newcommand{\BuiltInTok}[1]{{#1}}
    \newcommand{\ExtensionTok}[1]{{#1}}
    \newcommand{\PreprocessorTok}[1]{\textcolor[rgb]{0.74,0.48,0.00}{{#1}}}
    \newcommand{\AttributeTok}[1]{\textcolor[rgb]{0.49,0.56,0.16}{{#1}}}
    \newcommand{\InformationTok}[1]{\textcolor[rgb]{0.38,0.63,0.69}{\textbf{\textit{{#1}}}}}
    \newcommand{\WarningTok}[1]{\textcolor[rgb]{0.38,0.63,0.69}{\textbf{\textit{{#1}}}}}
    
    
    % Define a nice break command that doesn't care if a line doesn't already
    % exist.
    \def\br{\hspace*{\fill} \\* }
    % Math Jax compatability definitions
    \def\gt{>}
    \def\lt{<}
    % Document parameters
    \title{Wongnophadol\_Atit\_Assignment6}
    
    
    

    % Pygments definitions
    
\makeatletter
\def\PY@reset{\let\PY@it=\relax \let\PY@bf=\relax%
    \let\PY@ul=\relax \let\PY@tc=\relax%
    \let\PY@bc=\relax \let\PY@ff=\relax}
\def\PY@tok#1{\csname PY@tok@#1\endcsname}
\def\PY@toks#1+{\ifx\relax#1\empty\else%
    \PY@tok{#1}\expandafter\PY@toks\fi}
\def\PY@do#1{\PY@bc{\PY@tc{\PY@ul{%
    \PY@it{\PY@bf{\PY@ff{#1}}}}}}}
\def\PY#1#2{\PY@reset\PY@toks#1+\relax+\PY@do{#2}}

\expandafter\def\csname PY@tok@w\endcsname{\def\PY@tc##1{\textcolor[rgb]{0.73,0.73,0.73}{##1}}}
\expandafter\def\csname PY@tok@c\endcsname{\let\PY@it=\textit\def\PY@tc##1{\textcolor[rgb]{0.25,0.50,0.50}{##1}}}
\expandafter\def\csname PY@tok@cp\endcsname{\def\PY@tc##1{\textcolor[rgb]{0.74,0.48,0.00}{##1}}}
\expandafter\def\csname PY@tok@k\endcsname{\let\PY@bf=\textbf\def\PY@tc##1{\textcolor[rgb]{0.00,0.50,0.00}{##1}}}
\expandafter\def\csname PY@tok@kp\endcsname{\def\PY@tc##1{\textcolor[rgb]{0.00,0.50,0.00}{##1}}}
\expandafter\def\csname PY@tok@kt\endcsname{\def\PY@tc##1{\textcolor[rgb]{0.69,0.00,0.25}{##1}}}
\expandafter\def\csname PY@tok@o\endcsname{\def\PY@tc##1{\textcolor[rgb]{0.40,0.40,0.40}{##1}}}
\expandafter\def\csname PY@tok@ow\endcsname{\let\PY@bf=\textbf\def\PY@tc##1{\textcolor[rgb]{0.67,0.13,1.00}{##1}}}
\expandafter\def\csname PY@tok@nb\endcsname{\def\PY@tc##1{\textcolor[rgb]{0.00,0.50,0.00}{##1}}}
\expandafter\def\csname PY@tok@nf\endcsname{\def\PY@tc##1{\textcolor[rgb]{0.00,0.00,1.00}{##1}}}
\expandafter\def\csname PY@tok@nc\endcsname{\let\PY@bf=\textbf\def\PY@tc##1{\textcolor[rgb]{0.00,0.00,1.00}{##1}}}
\expandafter\def\csname PY@tok@nn\endcsname{\let\PY@bf=\textbf\def\PY@tc##1{\textcolor[rgb]{0.00,0.00,1.00}{##1}}}
\expandafter\def\csname PY@tok@ne\endcsname{\let\PY@bf=\textbf\def\PY@tc##1{\textcolor[rgb]{0.82,0.25,0.23}{##1}}}
\expandafter\def\csname PY@tok@nv\endcsname{\def\PY@tc##1{\textcolor[rgb]{0.10,0.09,0.49}{##1}}}
\expandafter\def\csname PY@tok@no\endcsname{\def\PY@tc##1{\textcolor[rgb]{0.53,0.00,0.00}{##1}}}
\expandafter\def\csname PY@tok@nl\endcsname{\def\PY@tc##1{\textcolor[rgb]{0.63,0.63,0.00}{##1}}}
\expandafter\def\csname PY@tok@ni\endcsname{\let\PY@bf=\textbf\def\PY@tc##1{\textcolor[rgb]{0.60,0.60,0.60}{##1}}}
\expandafter\def\csname PY@tok@na\endcsname{\def\PY@tc##1{\textcolor[rgb]{0.49,0.56,0.16}{##1}}}
\expandafter\def\csname PY@tok@nt\endcsname{\let\PY@bf=\textbf\def\PY@tc##1{\textcolor[rgb]{0.00,0.50,0.00}{##1}}}
\expandafter\def\csname PY@tok@nd\endcsname{\def\PY@tc##1{\textcolor[rgb]{0.67,0.13,1.00}{##1}}}
\expandafter\def\csname PY@tok@s\endcsname{\def\PY@tc##1{\textcolor[rgb]{0.73,0.13,0.13}{##1}}}
\expandafter\def\csname PY@tok@sd\endcsname{\let\PY@it=\textit\def\PY@tc##1{\textcolor[rgb]{0.73,0.13,0.13}{##1}}}
\expandafter\def\csname PY@tok@si\endcsname{\let\PY@bf=\textbf\def\PY@tc##1{\textcolor[rgb]{0.73,0.40,0.53}{##1}}}
\expandafter\def\csname PY@tok@se\endcsname{\let\PY@bf=\textbf\def\PY@tc##1{\textcolor[rgb]{0.73,0.40,0.13}{##1}}}
\expandafter\def\csname PY@tok@sr\endcsname{\def\PY@tc##1{\textcolor[rgb]{0.73,0.40,0.53}{##1}}}
\expandafter\def\csname PY@tok@ss\endcsname{\def\PY@tc##1{\textcolor[rgb]{0.10,0.09,0.49}{##1}}}
\expandafter\def\csname PY@tok@sx\endcsname{\def\PY@tc##1{\textcolor[rgb]{0.00,0.50,0.00}{##1}}}
\expandafter\def\csname PY@tok@m\endcsname{\def\PY@tc##1{\textcolor[rgb]{0.40,0.40,0.40}{##1}}}
\expandafter\def\csname PY@tok@gh\endcsname{\let\PY@bf=\textbf\def\PY@tc##1{\textcolor[rgb]{0.00,0.00,0.50}{##1}}}
\expandafter\def\csname PY@tok@gu\endcsname{\let\PY@bf=\textbf\def\PY@tc##1{\textcolor[rgb]{0.50,0.00,0.50}{##1}}}
\expandafter\def\csname PY@tok@gd\endcsname{\def\PY@tc##1{\textcolor[rgb]{0.63,0.00,0.00}{##1}}}
\expandafter\def\csname PY@tok@gi\endcsname{\def\PY@tc##1{\textcolor[rgb]{0.00,0.63,0.00}{##1}}}
\expandafter\def\csname PY@tok@gr\endcsname{\def\PY@tc##1{\textcolor[rgb]{1.00,0.00,0.00}{##1}}}
\expandafter\def\csname PY@tok@ge\endcsname{\let\PY@it=\textit}
\expandafter\def\csname PY@tok@gs\endcsname{\let\PY@bf=\textbf}
\expandafter\def\csname PY@tok@gp\endcsname{\let\PY@bf=\textbf\def\PY@tc##1{\textcolor[rgb]{0.00,0.00,0.50}{##1}}}
\expandafter\def\csname PY@tok@go\endcsname{\def\PY@tc##1{\textcolor[rgb]{0.53,0.53,0.53}{##1}}}
\expandafter\def\csname PY@tok@gt\endcsname{\def\PY@tc##1{\textcolor[rgb]{0.00,0.27,0.87}{##1}}}
\expandafter\def\csname PY@tok@err\endcsname{\def\PY@bc##1{\setlength{\fboxsep}{0pt}\fcolorbox[rgb]{1.00,0.00,0.00}{1,1,1}{\strut ##1}}}
\expandafter\def\csname PY@tok@kc\endcsname{\let\PY@bf=\textbf\def\PY@tc##1{\textcolor[rgb]{0.00,0.50,0.00}{##1}}}
\expandafter\def\csname PY@tok@kd\endcsname{\let\PY@bf=\textbf\def\PY@tc##1{\textcolor[rgb]{0.00,0.50,0.00}{##1}}}
\expandafter\def\csname PY@tok@kn\endcsname{\let\PY@bf=\textbf\def\PY@tc##1{\textcolor[rgb]{0.00,0.50,0.00}{##1}}}
\expandafter\def\csname PY@tok@kr\endcsname{\let\PY@bf=\textbf\def\PY@tc##1{\textcolor[rgb]{0.00,0.50,0.00}{##1}}}
\expandafter\def\csname PY@tok@bp\endcsname{\def\PY@tc##1{\textcolor[rgb]{0.00,0.50,0.00}{##1}}}
\expandafter\def\csname PY@tok@fm\endcsname{\def\PY@tc##1{\textcolor[rgb]{0.00,0.00,1.00}{##1}}}
\expandafter\def\csname PY@tok@vc\endcsname{\def\PY@tc##1{\textcolor[rgb]{0.10,0.09,0.49}{##1}}}
\expandafter\def\csname PY@tok@vg\endcsname{\def\PY@tc##1{\textcolor[rgb]{0.10,0.09,0.49}{##1}}}
\expandafter\def\csname PY@tok@vi\endcsname{\def\PY@tc##1{\textcolor[rgb]{0.10,0.09,0.49}{##1}}}
\expandafter\def\csname PY@tok@vm\endcsname{\def\PY@tc##1{\textcolor[rgb]{0.10,0.09,0.49}{##1}}}
\expandafter\def\csname PY@tok@sa\endcsname{\def\PY@tc##1{\textcolor[rgb]{0.73,0.13,0.13}{##1}}}
\expandafter\def\csname PY@tok@sb\endcsname{\def\PY@tc##1{\textcolor[rgb]{0.73,0.13,0.13}{##1}}}
\expandafter\def\csname PY@tok@sc\endcsname{\def\PY@tc##1{\textcolor[rgb]{0.73,0.13,0.13}{##1}}}
\expandafter\def\csname PY@tok@dl\endcsname{\def\PY@tc##1{\textcolor[rgb]{0.73,0.13,0.13}{##1}}}
\expandafter\def\csname PY@tok@s2\endcsname{\def\PY@tc##1{\textcolor[rgb]{0.73,0.13,0.13}{##1}}}
\expandafter\def\csname PY@tok@sh\endcsname{\def\PY@tc##1{\textcolor[rgb]{0.73,0.13,0.13}{##1}}}
\expandafter\def\csname PY@tok@s1\endcsname{\def\PY@tc##1{\textcolor[rgb]{0.73,0.13,0.13}{##1}}}
\expandafter\def\csname PY@tok@mb\endcsname{\def\PY@tc##1{\textcolor[rgb]{0.40,0.40,0.40}{##1}}}
\expandafter\def\csname PY@tok@mf\endcsname{\def\PY@tc##1{\textcolor[rgb]{0.40,0.40,0.40}{##1}}}
\expandafter\def\csname PY@tok@mh\endcsname{\def\PY@tc##1{\textcolor[rgb]{0.40,0.40,0.40}{##1}}}
\expandafter\def\csname PY@tok@mi\endcsname{\def\PY@tc##1{\textcolor[rgb]{0.40,0.40,0.40}{##1}}}
\expandafter\def\csname PY@tok@il\endcsname{\def\PY@tc##1{\textcolor[rgb]{0.40,0.40,0.40}{##1}}}
\expandafter\def\csname PY@tok@mo\endcsname{\def\PY@tc##1{\textcolor[rgb]{0.40,0.40,0.40}{##1}}}
\expandafter\def\csname PY@tok@ch\endcsname{\let\PY@it=\textit\def\PY@tc##1{\textcolor[rgb]{0.25,0.50,0.50}{##1}}}
\expandafter\def\csname PY@tok@cm\endcsname{\let\PY@it=\textit\def\PY@tc##1{\textcolor[rgb]{0.25,0.50,0.50}{##1}}}
\expandafter\def\csname PY@tok@cpf\endcsname{\let\PY@it=\textit\def\PY@tc##1{\textcolor[rgb]{0.25,0.50,0.50}{##1}}}
\expandafter\def\csname PY@tok@c1\endcsname{\let\PY@it=\textit\def\PY@tc##1{\textcolor[rgb]{0.25,0.50,0.50}{##1}}}
\expandafter\def\csname PY@tok@cs\endcsname{\let\PY@it=\textit\def\PY@tc##1{\textcolor[rgb]{0.25,0.50,0.50}{##1}}}

\def\PYZbs{\char`\\}
\def\PYZus{\char`\_}
\def\PYZob{\char`\{}
\def\PYZcb{\char`\}}
\def\PYZca{\char`\^}
\def\PYZam{\char`\&}
\def\PYZlt{\char`\<}
\def\PYZgt{\char`\>}
\def\PYZsh{\char`\#}
\def\PYZpc{\char`\%}
\def\PYZdl{\char`\$}
\def\PYZhy{\char`\-}
\def\PYZsq{\char`\'}
\def\PYZdq{\char`\"}
\def\PYZti{\char`\~}
% for compatibility with earlier versions
\def\PYZat{@}
\def\PYZlb{[}
\def\PYZrb{]}
\makeatother


    % Exact colors from NB
    \definecolor{incolor}{rgb}{0.0, 0.0, 0.5}
    \definecolor{outcolor}{rgb}{0.545, 0.0, 0.0}



    
    % Prevent overflowing lines due to hard-to-break entities
    \sloppy 
    % Setup hyperref package
    \hypersetup{
      breaklinks=true,  % so long urls are correctly broken across lines
      colorlinks=true,
      urlcolor=urlcolor,
      linkcolor=linkcolor,
      citecolor=citecolor,
      }
    % Slightly bigger margins than the latex defaults
    
    \geometry{verbose,tmargin=1in,bmargin=1in,lmargin=1in,rmargin=1in}
    
    

    \begin{document}
    
    
    \maketitle
    
    

    
    \subsection{Atit Wongnophadol}\label{atit-wongnophadol}

    \section{A Simple Hash Table with Linear
Probing}\label{a-simple-hash-table-with-linear-probing}

In this exercise, your task is to implement a variant of a hash table.
Several simplifications will make this task easier. First, your table
will have a fixed size; it will not be capable of expanding to fit more
data. Your table will only accept strings as keys, though values may be
any Python object. Finally, you will use linear probing to resolve
collisions.

Create a class, MyTable, with the following properties:

\textbf{Table:} Your table will have a fixed size, which you should pass
in as a parameter to the initializer. Specifically, you should create a
list to store keys (named keys) and a list to store values (named
values). All items in these lists should initially be set to the object
None.

\textbf{Keys and Values:} The keys to your table will be strings. Values
may be any python object.

\textbf{Hashcode:} Your class should convert each character in a key to
its unicode code point (use python's ord function) and then simply sum
them together.

\textbf{Compression function:} To ensure the results of your hashcode
falls in the right range, use the mod operator (\%) with the size of the
hash table.

\textbf{Collision resolution:} You will use linear probing to resolve
collisions. If a particular location in the table is filled, you move
forward one space until an empty location is found. If you reach the end
of the table, you cycle back to index 0.

\textbf{Deletions:} As with any open addressing system, deletions must
be executed with care. Finding one item A may rely on the fact that item
B was in a particular location when A was inserted. To get around this
problem, you should store three types of objects in your list of keys.
The object None indicates that a space has never been used. The special
string ``deleted'' indicates that the space was used before but is now
available. All other strings represent keys that have been stored in the
table.

Inside your MyTable class, you must implement the following methods:

\begin{itemize}
\item
  \_\_setitem\_\_(key, value) - insert the given key-value pair into the
  table. If the given key is already in the table, replace the old value
  with the new value.
\item
  \_\_getitem\_\_(key) - get the value that corresponds to the given key
  in the table.
\item
  \_\_delitem\_\_(key) - remove the given key and its corresponding
  value from the table. Replace both with the special string
  ``deleted''.
\end{itemize}

Note that these are magic methods that should not be accessed directly,
but will be called when indexing instances of your class with square
brackets

In case \_\_getitem\_\_ is called with a key that is not in the table,
return the string. ``Key not in table.''

Additionally, you should only access your keys list one index at a time
and avoid looping through all items in the list whenever possible. This
also means that you should not use operators like \emph{in} that
implicitely loop through all items in your list.

The following code demonstrates the proper use of the MyTable class.
Make sure that your class replicates this behavior exactly.

    \begin{Verbatim}[commandchars=\\\{\}]
{\color{incolor}In [{\color{incolor}1}]:} \PY{k}{class} \PY{n+nc}{MyTable}\PY{p}{(}\PY{p}{)}\PY{p}{:}
             
            \PY{k}{def} \PY{n+nf}{\PYZus{}\PYZus{}init\PYZus{}\PYZus{}}\PY{p}{(}\PY{n+nb+bp}{self}\PY{p}{,} \PY{n}{size}\PY{p}{)}\PY{p}{:}
                \PY{n+nb+bp}{self}\PY{o}{.}\PY{n}{\PYZus{}n} \PY{o}{=} \PY{n}{size}
                \PY{n+nb+bp}{self}\PY{o}{.}\PY{n}{keys} \PY{o}{=} \PY{n}{size}\PY{o}{*}\PY{p}{[}\PY{k+kc}{None}\PY{p}{]}
                \PY{n+nb+bp}{self}\PY{o}{.}\PY{n}{values} \PY{o}{=} \PY{n}{size}\PY{o}{*}\PY{p}{[}\PY{k+kc}{None}\PY{p}{]}
             
            \PY{k}{def} \PY{n+nf}{\PYZus{}\PYZus{}setitem\PYZus{}\PYZus{}}\PY{p}{(}\PY{n+nb+bp}{self}\PY{p}{,} \PY{n}{k}\PY{p}{,} \PY{n}{v}\PY{p}{)}\PY{p}{:}
                \PY{n+nb+bp}{self}\PY{o}{.}\PY{n}{\PYZus{}setvalue}\PY{p}{(}\PY{n}{k}\PY{p}{,} \PY{n}{v}\PY{p}{)}
            
            \PY{k}{def} \PY{n+nf}{\PYZus{}\PYZus{}getitem\PYZus{}\PYZus{}}\PY{p}{(}\PY{n+nb+bp}{self}\PY{p}{,} \PY{n}{k}\PY{p}{)}\PY{p}{:}
                \PY{k}{return} \PY{n+nb+bp}{self}\PY{o}{.}\PY{n}{\PYZus{}getvalue}\PY{p}{(}\PY{n}{k}\PY{p}{)}
            
            \PY{k}{def} \PY{n+nf}{\PYZus{}\PYZus{}delitem\PYZus{}\PYZus{}}\PY{p}{(}\PY{n+nb+bp}{self}\PY{p}{,} \PY{n}{k}\PY{p}{)}\PY{p}{:}
                \PY{n+nb+bp}{self}\PY{o}{.}\PY{n}{\PYZus{}delvalue}\PY{p}{(}\PY{n}{k}\PY{p}{)}
                
            \PY{k}{def} \PY{n+nf}{\PYZus{}hash\PYZus{}function}\PY{p}{(}\PY{n+nb+bp}{self}\PY{p}{,} \PY{n}{k}\PY{p}{)}\PY{p}{:}
                \PY{n}{i} \PY{o}{=} \PY{l+m+mi}{0}
                \PY{n}{hashed\PYZus{}k} \PY{o}{=} \PY{l+m+mi}{0}
                \PY{k}{while} \PY{n}{i} \PY{o}{\PYZlt{}} \PY{n+nb}{len}\PY{p}{(}\PY{n}{k}\PY{p}{)}\PY{p}{:}
                    \PY{n}{hashed\PYZus{}k}\PY{o}{+}\PY{o}{=}\PY{n+nb}{ord}\PY{p}{(}\PY{n}{k}\PY{p}{[}\PY{n}{i}\PY{p}{]}\PY{p}{)}
                    \PY{n}{i}\PY{o}{+}\PY{o}{=}\PY{l+m+mi}{1}
                \PY{k}{return} \PY{n}{hashed\PYZus{}k}
            
            \PY{k}{def} \PY{n+nf}{\PYZus{}compression\PYZus{}function}\PY{p}{(}\PY{n+nb+bp}{self}\PY{p}{,} \PY{n}{k}\PY{p}{)}\PY{p}{:}
                \PY{k}{return} \PY{n}{k}\PY{o}{\PYZpc{}}\PY{k}{self}.\PYZus{}n
            
            \PY{k}{def} \PY{n+nf}{\PYZus{}collision\PYZus{}solution}\PY{p}{(}\PY{n+nb+bp}{self}\PY{p}{,} \PY{n}{k}\PY{p}{)}\PY{p}{:}
                \PY{k}{return} \PY{p}{(}\PY{n}{k}\PY{o}{+}\PY{l+m+mi}{1}\PY{p}{)}\PY{o}{\PYZpc{}}\PY{k}{self}.\PYZus{}n
           
            \PY{k}{def} \PY{n+nf}{\PYZus{}setvalue}\PY{p}{(}\PY{n+nb+bp}{self}\PY{p}{,} \PY{n}{k}\PY{p}{,} \PY{n}{v}\PY{p}{)}\PY{p}{:} 
                
                \PY{n}{position} \PY{o}{=} \PY{n+nb+bp}{self}\PY{o}{.}\PY{n}{\PYZus{}compression\PYZus{}function}\PY{p}{(}\PY{n+nb+bp}{self}\PY{o}{.}\PY{n}{\PYZus{}hash\PYZus{}function}\PY{p}{(}\PY{n}{k}\PY{p}{)}\PY{p}{)}
                
                \PY{k}{if} \PY{n+nb+bp}{self}\PY{o}{.}\PY{n}{keys}\PY{p}{[}\PY{n}{position}\PY{p}{]} \PY{o}{==} \PY{k+kc}{None}\PY{p}{:}
                    \PY{n+nb+bp}{self}\PY{o}{.}\PY{n}{keys}\PY{p}{[}\PY{n}{position}\PY{p}{]} \PY{o}{=} \PY{n}{k}
                    \PY{n+nb+bp}{self}\PY{o}{.}\PY{n}{values}\PY{p}{[}\PY{n}{position}\PY{p}{]} \PY{o}{=} \PY{n}{v}
                \PY{k}{else}\PY{p}{:}
                    \PY{k}{if} \PY{n+nb+bp}{self}\PY{o}{.}\PY{n}{keys}\PY{p}{[}\PY{n}{position}\PY{p}{]} \PY{o}{==} \PY{n}{k}\PY{p}{:}
                        \PY{n+nb+bp}{self}\PY{o}{.}\PY{n}{values}\PY{p}{[}\PY{n}{position}\PY{p}{]} \PY{o}{=} \PY{n}{v}  \PY{c+c1}{\PYZsh{}replace}
                    \PY{k}{else}\PY{p}{:}
                        \PY{n}{position} \PY{o}{=} \PY{n+nb+bp}{self}\PY{o}{.}\PY{n}{\PYZus{}collision\PYZus{}solution}\PY{p}{(}\PY{n}{position}\PY{p}{)}
                        \PY{k}{while} \PY{n+nb+bp}{self}\PY{o}{.}\PY{n}{keys}\PY{p}{[}\PY{n}{position}\PY{p}{]} \PY{o}{!=} \PY{k+kc}{None} \PY{o+ow}{and} \PY{n+nb+bp}{self}\PY{o}{.}\PY{n}{keys}\PY{p}{[}\PY{n}{position}\PY{p}{]} \PY{o}{!=} \PY{n}{k}\PY{p}{:}
                            \PY{n}{position} \PY{o}{=} \PY{n+nb+bp}{self}\PY{o}{.}\PY{n}{\PYZus{}collision\PYZus{}solution}\PY{p}{(}\PY{n}{position}\PY{p}{)}
                        \PY{k}{if} \PY{n+nb+bp}{self}\PY{o}{.}\PY{n}{keys}\PY{p}{[}\PY{n}{position}\PY{p}{]} \PY{o}{==} \PY{k+kc}{None}\PY{p}{:}
                            \PY{n+nb+bp}{self}\PY{o}{.}\PY{n}{keys}\PY{p}{[}\PY{n}{position}\PY{p}{]} \PY{o}{=} \PY{n}{k}
                            \PY{n+nb+bp}{self}\PY{o}{.}\PY{n}{values}\PY{p}{[}\PY{n}{position}\PY{p}{]} \PY{o}{=} \PY{n}{v}
                        \PY{k}{else}\PY{p}{:}
                            \PY{n+nb+bp}{self}\PY{o}{.}\PY{n}{values}\PY{p}{[}\PY{n}{position}\PY{p}{]} \PY{o}{=} \PY{n}{v}
             
            \PY{k}{def} \PY{n+nf}{\PYZus{}getvalue}\PY{p}{(}\PY{n+nb+bp}{self}\PY{p}{,} \PY{n}{k}\PY{p}{)}\PY{p}{:}
                
                \PY{n}{initial\PYZus{}position} \PY{o}{=} \PY{n+nb+bp}{self}\PY{o}{.}\PY{n}{\PYZus{}compression\PYZus{}function}\PY{p}{(}\PY{n+nb+bp}{self}\PY{o}{.}\PY{n}{\PYZus{}hash\PYZus{}function}\PY{p}{(}\PY{n}{k}\PY{p}{)}\PY{p}{)}
                
                \PY{n}{value} \PY{o}{=} \PY{k+kc}{None}
                \PY{n}{stop} \PY{o}{=} \PY{k+kc}{False}
                \PY{n}{found} \PY{o}{=} \PY{k+kc}{False}
                \PY{n}{position} \PY{o}{=} \PY{n}{initial\PYZus{}position}
                \PY{k}{while} \PY{n+nb+bp}{self}\PY{o}{.}\PY{n}{keys}\PY{p}{[}\PY{n}{position}\PY{p}{]} \PY{o}{!=} \PY{k+kc}{None} \PY{o+ow}{and} \PY{o+ow}{not} \PY{n}{found} \PY{o+ow}{and} \PY{o+ow}{not} \PY{n}{stop}\PY{p}{:}
                    \PY{k}{if} \PY{n+nb+bp}{self}\PY{o}{.}\PY{n}{keys}\PY{p}{[}\PY{n}{position}\PY{p}{]} \PY{o}{==} \PY{n}{k}\PY{p}{:}
                        \PY{n}{found} \PY{o}{=} \PY{k+kc}{True}
                        \PY{n}{value} \PY{o}{=} \PY{n+nb+bp}{self}\PY{o}{.}\PY{n}{values}\PY{p}{[}\PY{n}{position}\PY{p}{]}
                    \PY{k}{else}\PY{p}{:}
                        \PY{n}{position}\PY{o}{=}\PY{n+nb+bp}{self}\PY{o}{.}\PY{n}{\PYZus{}collision\PYZus{}solution}\PY{p}{(}\PY{n}{position}\PY{p}{)}
                        \PY{k}{if} \PY{n}{position} \PY{o}{==} \PY{n}{initial\PYZus{}position}\PY{p}{:}
                            \PY{n}{stop} \PY{o}{=} \PY{k+kc}{True}
         
                \PY{k}{if} \PY{n}{value} \PY{o+ow}{is} \PY{k+kc}{None}\PY{p}{:}
                    \PY{k}{return} \PY{l+s+s2}{\PYZdq{}}\PY{l+s+s2}{Key not in table}\PY{l+s+s2}{\PYZdq{}}
                \PY{k}{else}\PY{p}{:}
                    \PY{k}{return} \PY{n}{value}
            
            \PY{k}{def} \PY{n+nf}{\PYZus{}delvalue}\PY{p}{(}\PY{n+nb+bp}{self}\PY{p}{,} \PY{n}{k}\PY{p}{)}\PY{p}{:}
                
                \PY{n}{initial\PYZus{}position} \PY{o}{=} \PY{n+nb+bp}{self}\PY{o}{.}\PY{n}{\PYZus{}compression\PYZus{}function}\PY{p}{(}\PY{n+nb+bp}{self}\PY{o}{.}\PY{n}{\PYZus{}hash\PYZus{}function}\PY{p}{(}\PY{n}{k}\PY{p}{)}\PY{p}{)}
                
                \PY{n}{stop} \PY{o}{=} \PY{k+kc}{False}
                \PY{n}{found} \PY{o}{=} \PY{k+kc}{False}
                \PY{n}{position} \PY{o}{=} \PY{n}{initial\PYZus{}position}
                \PY{k}{while} \PY{n+nb+bp}{self}\PY{o}{.}\PY{n}{keys}\PY{p}{[}\PY{n}{position}\PY{p}{]} \PY{o}{!=} \PY{k+kc}{None} \PY{o+ow}{and} \PY{o+ow}{not} \PY{n}{found} \PY{o+ow}{and} \PY{o+ow}{not} \PY{n}{stop}\PY{p}{:}
                    \PY{k}{if} \PY{n+nb+bp}{self}\PY{o}{.}\PY{n}{keys}\PY{p}{[}\PY{n}{position}\PY{p}{]} \PY{o}{==} \PY{n}{k}\PY{p}{:}
                        \PY{n}{found} \PY{o}{=} \PY{k+kc}{True}
                        \PY{n+nb+bp}{self}\PY{o}{.}\PY{n}{keys}\PY{p}{[}\PY{n}{position}\PY{p}{]} \PY{o}{=} \PY{l+s+s2}{\PYZdq{}}\PY{l+s+s2}{deleted}\PY{l+s+s2}{\PYZdq{}}
                        \PY{n+nb+bp}{self}\PY{o}{.}\PY{n}{values}\PY{p}{[}\PY{n}{position}\PY{p}{]} \PY{o}{=} \PY{l+s+s2}{\PYZdq{}}\PY{l+s+s2}{deleted}\PY{l+s+s2}{\PYZdq{}}
                    \PY{k}{else}\PY{p}{:}
                        \PY{n}{position}\PY{o}{=}\PY{n+nb+bp}{self}\PY{o}{.}\PY{n}{\PYZus{}collision\PYZus{}solution}\PY{p}{(}\PY{n}{position}\PY{p}{)}
                        \PY{k}{if} \PY{n}{position} \PY{o}{==} \PY{n}{initial\PYZus{}position}\PY{p}{:}
                            \PY{n}{stop} \PY{o}{=} \PY{k+kc}{True}
\end{Verbatim}


    \begin{Verbatim}[commandchars=\\\{\}]
{\color{incolor}In [{\color{incolor}2}]:} \PY{n}{m} \PY{o}{=} \PY{n}{MyTable}\PY{p}{(}\PY{l+m+mi}{5}\PY{p}{)}
        \PY{c+c1}{\PYZsh{} The following keys all hash to the same index.}
        \PY{n}{m}\PY{p}{[}\PY{l+s+s2}{\PYZdq{}}\PY{l+s+s2}{a}\PY{l+s+s2}{\PYZdq{}}\PY{p}{]} \PY{o}{=} \PY{l+s+s2}{\PYZdq{}}\PY{l+s+s2}{apple}\PY{l+s+s2}{\PYZdq{}}
        \PY{n}{m}\PY{p}{[}\PY{l+s+s2}{\PYZdq{}}\PY{l+s+s2}{f}\PY{l+s+s2}{\PYZdq{}}\PY{p}{]} \PY{o}{=} \PY{l+s+s2}{\PYZdq{}}\PY{l+s+s2}{butter}\PY{l+s+s2}{\PYZdq{}} 
        \PY{n}{m}\PY{p}{[}\PY{l+s+s2}{\PYZdq{}}\PY{l+s+s2}{k}\PY{l+s+s2}{\PYZdq{}}\PY{p}{]} \PY{o}{=} \PY{l+s+s2}{\PYZdq{}}\PY{l+s+s2}{yummy}\PY{l+s+s2}{\PYZdq{}}
        \PY{n+nb}{print}\PY{p}{(}\PY{l+s+s2}{\PYZdq{}}\PY{l+s+s2}{Current keys in m:}\PY{l+s+s2}{\PYZdq{}}\PY{p}{,} \PY{n}{m}\PY{o}{.}\PY{n}{keys}\PY{p}{)}
\end{Verbatim}


    \begin{Verbatim}[commandchars=\\\{\}]
Current keys in m: [None, None, 'a', 'f', 'k']

    \end{Verbatim}

    \begin{Verbatim}[commandchars=\\\{\}]
{\color{incolor}In [{\color{incolor}3}]:} \PY{c+c1}{\PYZsh{} \PYZdq{}p\PYZdq{} also hashes to the same place.}
        \PY{c+c1}{\PYZsh{} Your class should detect that it\PYZsq{}s not in the table}
        \PY{c+c1}{\PYZsh{} without scanning through the entire keys list.}
        \PY{n+nb}{print}\PY{p}{(}\PY{l+s+s2}{\PYZdq{}}\PY{l+s+s2}{m[}\PY{l+s+s2}{\PYZsq{}}\PY{l+s+s2}{p}\PY{l+s+s2}{\PYZsq{}}\PY{l+s+s2}{]:}\PY{l+s+s2}{\PYZdq{}}\PY{p}{,} \PY{n}{m}\PY{p}{[}\PY{l+s+s2}{\PYZdq{}}\PY{l+s+s2}{p}\PY{l+s+s2}{\PYZdq{}}\PY{p}{]}\PY{p}{)}
\end{Verbatim}


    \begin{Verbatim}[commandchars=\\\{\}]
m['p']: Key not in table

    \end{Verbatim}

    \begin{Verbatim}[commandchars=\\\{\}]
{\color{incolor}In [{\color{incolor}4}]:} \PY{c+c1}{\PYZsh{} We can access key \PYZdq{}k\PYZdq{}}
        \PY{n+nb}{print}\PY{p}{(}\PY{l+s+s2}{\PYZdq{}}\PY{l+s+s2}{m[}\PY{l+s+s2}{\PYZsq{}}\PY{l+s+s2}{k}\PY{l+s+s2}{\PYZsq{}}\PY{l+s+s2}{]:}\PY{l+s+s2}{\PYZdq{}}\PY{p}{,} \PY{n}{m}\PY{p}{[}\PY{l+s+s2}{\PYZdq{}}\PY{l+s+s2}{k}\PY{l+s+s2}{\PYZdq{}}\PY{p}{]}\PY{p}{)}
        \PY{c+c1}{\PYZsh{} Even if we remove \PYZdq{}f\PYZdq{}}
        \PY{k}{del} \PY{n}{m}\PY{p}{[}\PY{l+s+s2}{\PYZdq{}}\PY{l+s+s2}{f}\PY{l+s+s2}{\PYZdq{}}\PY{p}{]}
        \PY{n+nb}{print}\PY{p}{(}\PY{l+s+s2}{\PYZdq{}}\PY{l+s+s2}{m[}\PY{l+s+s2}{\PYZsq{}}\PY{l+s+s2}{k}\PY{l+s+s2}{\PYZsq{}}\PY{l+s+s2}{]:}\PY{l+s+s2}{\PYZdq{}}\PY{p}{,} \PY{n}{m}\PY{p}{[}\PY{l+s+s2}{\PYZdq{}}\PY{l+s+s2}{k}\PY{l+s+s2}{\PYZdq{}}\PY{p}{]}\PY{p}{)}
        \PY{n+nb}{print}\PY{p}{(}\PY{l+s+s2}{\PYZdq{}}\PY{l+s+s2}{Current keys in m:}\PY{l+s+s2}{\PYZdq{}}\PY{p}{,} \PY{n}{m}\PY{o}{.}\PY{n}{keys}\PY{p}{)}
\end{Verbatim}


    \begin{Verbatim}[commandchars=\\\{\}]
m['k']: yummy
m['k']: yummy
Current keys in m: [None, None, 'a', 'deleted', 'k']

    \end{Verbatim}

    \begin{Verbatim}[commandchars=\\\{\}]
{\color{incolor}In [{\color{incolor}7}]:} \PY{c+c1}{\PYZsh{} Even after removing \PYZdq{}f\PYZdq{}, we can overwrite \PYZdq{}k\PYZdq{}}
        \PY{n}{m}\PY{p}{[}\PY{l+s+s2}{\PYZdq{}}\PY{l+s+s2}{k}\PY{l+s+s2}{\PYZdq{}}\PY{p}{]} \PY{o}{=} \PY{l+s+s2}{\PYZdq{}}\PY{l+s+s2}{newstuff}\PY{l+s+s2}{\PYZdq{}}
        \PY{n+nb}{print}\PY{p}{(}\PY{l+s+s2}{\PYZdq{}}\PY{l+s+s2}{Current keys in m:}\PY{l+s+s2}{\PYZdq{}}\PY{p}{,} \PY{n}{m}\PY{o}{.}\PY{n}{keys}\PY{p}{)}
        \PY{n+nb}{print}\PY{p}{(}\PY{l+s+s2}{\PYZdq{}}\PY{l+s+s2}{Current values in m:}\PY{l+s+s2}{\PYZdq{}}\PY{p}{,} \PY{n}{m}\PY{o}{.}\PY{n}{values}\PY{p}{)}
\end{Verbatim}


    \begin{Verbatim}[commandchars=\\\{\}]
Current keys in m: [None, None, 'a', 'deleted', 'k']
Current values in m: [None, None, 'apple', 'deleted', 'newstuff']

    \end{Verbatim}


    % Add a bibliography block to the postdoc
    
    
    
    \end{document}
